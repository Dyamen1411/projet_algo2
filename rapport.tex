\documentclass[12pt, letterpaper]{report}
\usepackage[utf8]{inputenc}
\usepackage[ margin=1in]{geometry}
\usepackage{color}


\title{Rapport de projet d'algorithmique 2 : 
\\structures récursives, linéaires et binaires}
\author{KLAK Karolina et MASSIAS Antoine \thanks 
{Rapport réalisé avec laTex} }
\date {Janvier 2022}


\setlength{\parindent}{3em}
\setlength{\parskip}{1em}
%corps du document
\begin{document}
\begin{titlepage}
    \maketitle
\end{titlepage}


\tableofcontents

\chapter{Généralités}

Ce projet constitue l'aboutissement du projet
d'algorithmique 2, du semestre 3 de la licence 
d'informatique à l'UFR Sciences et Techniques de Rouen.


Le projet a été réalisé par A. Massias et K. Klak dans le cadre de leurs études.
Ce projet consiste en la réalisation, en C, d'un exécutable produisant sur la sortie
 standard une liste de mots partagés. En effet, cet exécutable doit,
  au terme de plusieurs noms de fichiers inscrits sur la ligne de commande de trier les mots figurant dans chacun d'eux, 
  et d'en conclure une liste de "mots" dits "partagés" entre les fichiers, tout en spécifiant leur motif et leur nombre total d'occurences.
  A titre informatif, le motif fait référence aux fichiers dans lesquels ledit mot apparaît.
  Il se compose d'une suite de 'x' ou '-' en focntion de la présence ou non du mot dans chacun des fichiers.
\section{Spécifications}
La liste dont il est question est triée selon trois filtres : 
\\-ordre décroissant du nombre de fichiers dans lesquels le mot figure,
\\-ordre décroissant des nombres totaux d'occurrences,
\\-ordre lexicographique croissant des mots.

Une ligne est réalisé pour chaque mot, avec le descriptif ci-dessus.
 Chaque liste est composée de dix lignes, limite fixée par défaut qu'il
  est possible de modifier par le biais d'options (cf. 1.2 Options).
Une autre limitation fixée par défaut : le nombre de caractères de chacun
des mots étant pris en compte. En effet, par défaut seuls les 63 premeirs
caractères d'un mot sont pris en compte. Au-delà de cette limite, le mot est 
coupé. 
\\ \\ \\ \\ \\ \\ \\ 
\textcolor{blue}{
\emph{Exemple : 
L'exécutable produit par défaut les lignes de textes suivantes : }}
\\\textbf{\$} ./ws textes/tartuffe.txt textes/lesmiserables.txt
\\./ws: Word from 'textes/lesmiserables.txt' at line 33239 cut: 
'paroisse-meurs-de-faim-si-tu-as-du-feu-meurs-de-froid-si-tu-as-...'.
\\./ws: Word from 'textes/lesmiserables.txt' at line 62928 cut: 
'--L'agent-de-police-Ja-vert-a-ete-trouve-noye-sous-un-bateau-du...'.
\\xx      22021   de
\\xx      15256   la
\\xx      13062   et
\\xx      11685   a
\\xx      11194   le
\\xx      7252    les
\\xx      6114    un
\\xx      5667    que
\\xx      5444    il
\\xx      5375    qui

\emph{
Il y a précision si des mots ont été coupés.
Ainsi comme convenu seules 10 lignes de textes sont affichées.
\\Le motif ici est 'xx' pour chacun des mots de la liste : cela signifie 
 que les mots apparaissent dans les deux fichiers. \\Le nombre qui suit 
 indique le nombre d'occurrences totales, dans tous les fichiers.
 \\ Enfin, en toute logique, la dernière chaine de caractères
  de la ligne correspond au mot, entier ou coupé.}

\section{Options}
En totalité, cinq options coexistent : deux d'entre elles concernent 
le nombre de lignes de mots, une concernant la longueur maximale du 
préfixe du mot, et enfin deux sur la notion du mot.
\\En outre : 
\\- '-t' : limite du nombre de lignes de textes ( limite du nombre de 
mots à produire)
\\- '-s' : production de lignes supplémentaires de tous les possibles mots
dont le motif et le nombre d'occurrences sont égaux à ceux du dernier 
des mots associés à la limite
\\- '-i' : fixer la longueur maximale du préfixe de chaque mot
\\- '-p' : permet de ne plus considérer 
une marque de ponctuation comme un "mot"
\\- '-u' : permet de convertir un caractère correspondant à une lettre 
minuscule en un caractère de lettre majuscule
\\ \\ \\ \\

\textcolor{blue}{
\emph{Exemple : ci-dessous les lignes de textes en sortie avec 
l'utilisation des 5 options.} }

\noindent \textbf{\$} ./ws -t 12 -s -i 5 -p -u textes/toujoursetjamais.txt  textes/lecid.txt textes/lesmiserables.txt textes/lavare.txt
\\...
\\./ws: Word from 'textes/lecid.txt' at line 34 cut: 'TEMPS...'.
\\./ws: Word from 'textes/lecid.txt' at line 34 cut: 'VAILL...'.
\\xxxx    18020   LA
\\xxxx    14920   ET
\\xxxx    13995   A
\\xxxx    13407   LE
\\xxxx    11824   L
\\xxxx    8742    UN
\\xxxx    8396    LES
\\xxxx    6531    D
\\xxxx    6246    UNE
\\xxxx    6169    EST
\\xxxx    5943    QUI
\\xxxx    5671    DANS

\end{document}